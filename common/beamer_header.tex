%-------------------------------------------------------------------------------------[type]
\documentclass[aspectratio=169]{beamer}
\usepackage[T1]{fontenc}
%-------------------------------------------------------------------------------------[移除导航]
\setbeamertemplate{navigation symbols}{}
%-------------------------------------------------------------------------------------[theme]
\usetheme{Madrid}
\usecolortheme{beaver}
%-------------------------------------------------------------------------------------[date]
\usepackage[USenglish,UKenglish,french,spanish,italian]{babel}
\usepackage[useregional]{datetime2}
%-------------------------------------------------------------------------------------[pic]
\usepackage{graphicx}
%-------------------------------------------------------------------------------------[#subfigure#]
\usepackage{subcaption}
\newcommand{\subfigwid}{0.24}
%-------------------------------------------------------------------------------------[listings]
\usepackage{listings}
\usepackage{xcolor}
\definecolor{codegreen}{rgb}{0,0.6,0}
\definecolor{codegray}{rgb}{0.5,0.5,0.5}
\definecolor{codepurple}{rgb}{0.58,0,0.82}
\definecolor{backcolour}{rgb}{0.95,0.95,0.92}

\lstdefinestyle{mystyle}{
    backgroundcolor=\color{backcolour},   
    commentstyle=\color{codegreen},
    keywordstyle=\color{magenta},
    numberstyle=\tiny\color{codegray},
    stringstyle=\color{codepurple},
    basicstyle=\ttfamily\fontsize{9pt}{11pt}\selectfont,
    breakatwhitespace=false,         
    breaklines=true,                 
    captionpos=b,                    
    keepspaces=true,                 
    numbers=left,                    
    numbersep=5pt,                  
    showspaces=false,                
    showstringspaces=false,
    showtabs=false,                  
    tabsize=2
}
\lstset{style=mystyle}
%-----------------------------------------------------------[multicol]
\usepackage{multicol}
%-------------------------------------------------------------------------------------[flowchart]
\usepackage{tikz}
\usetikzlibrary{shapes.geometric, arrows}
\tikzstyle{startstop} = [rectangle, rounded corners, minimum width=3cm, minimum height=1cm,text centered, draw=black, fill=red!30]
\tikzstyle{io} = [trapezium, trapezium left angle=70, trapezium right angle=110, minimum width=3cm, minimum height=1cm, text centered, draw=black, fill=blue!30]
\tikzstyle{process} = [rectangle, minimum width=3cm, minimum height=1cm, text centered, draw=black, fill=orange!30]
\tikzstyle{decision} = [diamond, minimum width=3cm, minimum height=1cm, text centered, draw=black, fill=green!30]
\tikzstyle{arrow} = [thick,->,>=stealth]
\usepackage{varwidth}
%-------------------------------------------------------------------------------------[comment]
\usepackage{comment}
%-------------------------------------------------------------------------------------[braket]
\usepackage{braket}
%-------------------------------------------------------------------------------------[#gif#]
\usepackage{animate}
%-------------------------------------------------------------------------------------[footfullcite]
\usepackage[style=chem-acs,doi=false,url=false]{biblatex}
%-----------------------------------------------------------------[多引用,]
\usepackage{fnpct}
\AdaptNote\footfullcite{oo+m}[footnote]{%
\setfnpct{dont-mess-around}%
\IfNoValueTF{#1}
{#NOTE{#3}}
{\IfNoValueTF{#2}{#NOTE[#1]{#3}}{#NOTE[#1][#2]{#3}}}%
}
%-----------------------------------------------------------------[字号]
% 7pt 字号,9pt baselineskip
\renewcommand{\footnotesize}{\fontsize{7pt}{9pt}\selectfont}
%-----------------------------------------------------------------[footnotemark]
% https://tex.stackexchange.com/questions/420162/footfullcite-not-showing-when-used-within-a-wrapfigure
\DeclareCiteCommand{\footfullcitetext}[\mkbibfootnotetext]
  {\usebibmacro{prenote}}
  {\usedriver
     {\DeclareNameAlias{sortname}{default}}
     {\thefield{entrytype}}}
  {\multicitedelim}
  {\usebibmacro{postnote}}

\newcounter{colcites}
%-------------------------------------------------------------------------------------[pvec]
\newcommand{\pvec}[1]{\vec{#1}\mkern2mu\vphantom{#1}}

\usepackage{multirow}
%-------------------------------------------------------------------------------------[积分]
\newcommand*{\diff}{\mathop{}\!\mathrm{d}}
%-------------------------------------------------------------------------------------[#超链接#]
% 放到 usepackage 最后
\usepackage{hyperref}

\hypersetup{
    colorlinks=true,
    linkcolor=blue,
    filecolor=magenta,
    urlcolor=cyan,
    citecolor=black,
}
\urlstyle{same}
%-------------------------------------------------------------------------------------[atbeginsection]
\AtBeginSection[]
{
  \begin{frame}
    \frametitle{Table of Contents}
    \tableofcontents[
        sectionstyle=show/shaded,
        subsectionstyle=hide,
        subsubsectionstyle=hide
    ]
  \end{frame}
}
\AtBeginSubsection[]
{
  \begin{frame}
    \frametitle{Table of Contents}
    \tableofcontents[
        currentsubsection,
        sectionstyle=show/hide,
        subsectionstyle=show/shaded/hide,
        subsubsectionstyle=hide
    ]
  \end{frame}
}
\AtBeginSubsubsection[]
{
  \begin{frame}
    \frametitle{Table of Contents}
    \tableofcontents[
        currentsubsection,
        sectionstyle=hide,
        subsectionstyle=show/hide/hide,
        subsubsectionstyle=show/shaded/hide/hide
    ]
  \end{frame}
}
%-------------------------------------------------------------------------------------[titleset]
\institute[SHTU] % (optional)
{
  School of Physical Science and Technology\\
  ShanghaiTech University
}
\date{\today}